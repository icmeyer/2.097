\documentclass[10pt,letter]{article}
	% basic article document class
	% use percent signs to make comments to yourself -- they will not show up.
\usepackage{amsmath}
\usepackage{amssymb}
	% packages that allow mathematical formatting
\usepackage{graphicx}
	% package that allows you to include graphics
\usepackage{setspace}
	% package that allows you to change spacing
\onehalfspacing
	% text become 1.5 spaced
\usepackage{fullpage}
	% package that specifies normal margins

\usepackage{hyperref}
\hypersetup{
    colorlinks=true,
    linkcolor=blue,
    filecolor=magenta,      
    urlcolor=cyan,
}
\urlstyle{same}
% \usepackage{siunitx}
% \sisetup{output-exponent-marker=\ensuremath{\mathrm{e}}}
\usepackage{float}
\usepackage[autostyle, english=american]{csquotes} \MakeOuterQuote{"} %This line brought to you by Travis
\usepackage{subfig}
\usepackage{pdflscape}

\begin{document}
	% line of code telling latex that your document is beginning


\title{2.097 Homework 0}

\author{Isaac Meyer\\
        Professor Wang}

	% Note: when you omit this command, the current dateis automatically included
 
\maketitle 
	% tells latex to follow your header (e.g., title, author) commands.

\section*{Overview}
All code used in this problem set is available on github at: \url{https://github.com/icmeyer/22.211/tree/master/pset6}\\

\section{Finite Difference}
\[ \frac{\delta u}{\delta t} = k \frac{\delta^2u}{\delta x^2} \]
We can write the second order derivative using finite differences. Assuming an even mesh spacing we use the following relations. 
\begin{align*}
    \frac{du}{dx}\Big|_{n+1/2}=\frac{\Delta u}{\Delta x} = \frac{u_{n+1}-u_n}{x_{n+1}-x_n} \\
    \frac{d^2u}{dx^2}\Big|_{n} = \frac{d}{dx} \frac{du}{dx}= \frac{u_{n+1}-2u_n+u_{n-1}}{\Delta x^2}
\end{align*}

This will allow us to write the equation as an ODE. Where $i$ represents the spatial index and $n$ the temporal index.
\[ \frac{u_i^{n+1}-u_i^n}{\Delta t} = k  \frac{u_{i+1}^n-2u_i^n+u_{i-1}^n}{\Delta x^2} \]
\[ u^{n+1} = \frac{k \Delta t}{\Delta x^2} (u_{i+1}^n-2u_i^n+u_{i-1}^n) + u_i^n \]
Writing in this in matrix form with the boundary conditions. Using $C = \frac{k \Delta t}{\Delta x^2}$

\begin{equation}
\label{eq:F}
\hspace{-2.5 cm}
\begin{bmatrix}
    u_1^{n+1}\\
    u_2^{n+1}\\
    \vdots \\
    u_i^{n+1} \\
    \vdots \\
    u_{N-1}^{n+1} \\
    u_N^{n+1} \\
\end{bmatrix}
 =
\begin{bmatrix}
    1 & 0 & 0 & 0 & 0 & 0 & 0\\
    C & -2C & C & 0 & 0 & 0 & 0\\
    0 & \ddots & \ddots & \ddots & 0 & 0 & 0\\
    0 & 0 & 0 & 0 & C & -2C & C\\
    0 & 0 & 0 & 0 & 0 & 0 & 1\\
\end{bmatrix}
\begin{bmatrix}
    u_{left} \\
    u_2^n \\
    \vdots \\
    u_i^n \\
    \vdots \\
    u_{N-1}^n \\
    u_{right} \\
\end{bmatrix}
+
\begin{bmatrix}
    0 \\
    u_2^n \\
    \vdots \\
    u_i^n \\
    \vdots \\
    u_{N-1}^n \\
    0 \\
\end{bmatrix}
\end{equation}
In order for this method to apply we must maintain a stability condition. This condition is given by:
\[ \frac{4D\Delta t}{\Delta x^2} < 2\]
My code will throw an error if this condition is not met. 





% Examples below

% Figure example
% \begin{figure}[H]
%     \centering
%     \includegraphics[width=1.05\textwidth]{figures/Figure_1-5.png}
%     \caption{Problem 5}
%     \label{fig:5}
% \end{figure}

% Equations (simple):
% \[  ]\


% Table Example
% \begin{table}[h]
% \begin{center}
% \caption{Numerical Results} \label{tab:fourfactors}
% \begin{tabular}{ |c|c|c|c|c| } 
% \hline
%  & $k_{eff}$ & Peak Fission Source Value & Location of Peak from center [cm] & Iterations\\ 
% \hline
% Problem 1 & 0.9546 & 1.5442 & 0.0 & 166 \\
% \hline
% Problem 2 & 1.1202 & 1.4627 & 0.0 & 137 \\
% \hline
% Problem 3 & 1.1202 & 1.4605 & 0.0 & 137 \\
% \hline
% Problem 4 & 0.9784 & 1.4329 & 112.0 & 166 \\
% \hline
% Problem 5 & 0.9733 & 1.1124 & 0.0 & 138 \\
% \hline
% \end{tabular}
% \end{center}
% \end{table}


% Code Example
% \begin{verbatim}
%     while True:
%         # Calculate new flux by solving H*phi_1=b_0
%         phinew = np.dot(hinv,b)]
%         # Calculate the new k
%         knew = np.sum(np.dot(fmat,phinew))/np.sum(np.dot(fmat,phiold))*kold
%         if (tolerance met?):
%             break
%         kold = knew
%         #Normalization step
%         phiold = normalize(phinew)
%         phioldsum = phinewsum
%         #Calculate the new fission source
%         b = (1/kold)*np.dot(fmat,phiold)
% \end{verbatim}

\end{document}
	% line of code telling latex that your document is ending. If you leave this out, you'll get an error
